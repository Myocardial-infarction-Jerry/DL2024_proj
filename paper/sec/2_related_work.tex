\section{Related Work}
\label{sec:related_work}

Fine-grained visual classification (FGVC) is an important task in computer vision, aiming to categorize images into highly specific and detailed subcategories, such as different species of birds, dogs, vehicles, etc. Compared to traditional image classification, FGVC requires identifying subtle visual differences, often located in small regions of the image, such as color, texture, shape, and patterns.

In the FGVC field, various methodologies have been proposed to address this challenge. These methods can be broadly categorized as follows:


\TODO{Provide a comprehensive review of existing literature on Fine-Grained Visual Classification (FGVC), including key methodologies, datasets, and evaluation metrics. Discuss the strengths and limitations of prior approaches, highlighting the gaps that this research aims to address.}
